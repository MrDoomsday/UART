\documentclass[a4paper,12pt]{article}

\usepackage[left=1.5cm,top=1.5cm,right=1.5cm,bottom=1.5cm,includehead,includefoot]{geometry}

\usepackage{cmap}
\usepackage[T2A]{fontenc}
\usepackage[utf8]{inputenc}
\usepackage[russian]{babel}
\usepackage[pdftex]{graphicx}
\usepackage{color}
\usepackage{ulem}
\usepackage{amsmath}
\usepackage{xcolor}
\usepackage{longtable}
\usepackage{hyperref}
\usepackage{import}
\usepackage{bytefield}
\usepackage{datetime}
\usepackage{float}
\usepackage{dirtree}
\usepackage{colortbl}
\usepackage{url}
\usepackage{multirow}
\graphicspath{{FIGURE/}}



\title{\Huge\textbf{Спецификация на ядро UART}}
\author{\huge RB}
\date{\today}

\pdfinfo{%
  /Title    (Спецификация на ядро UART)
  /Author   (RB)
  /Creator  (RB)
  /Producer (RB)
  /Subject  (IP-core specification)
  /Keywords (SystemVerilog, FPGA, UART, IP-core)
}


\begin{document}
\maketitle
\newpage


%
% --Список фигур---
%
\listoffigures
\addcontentsline{toc}{chapter}{\listfigurename}
\newpage

% --Список таблиц---
%
\listoftables
\addcontentsline{toc}{chapter}{\listtablename}
\newpage


%
%---Chapters----
%
%
% Introduction
%
\section{Введение}
\label{sec:introduction}

Данная спецификация описывает ядро UART, управление которым осуществляется при помощи интерфейса Avalon-MM или, в перспективе, AXI Lite (на данный момент находится в стадии разработки).

    \subsection{Интеграция}
    Изначально ядро предполагает интеграцию в QSYS САПР Quartus, но никто не запрещает добавлять его в любой RTL-код. Пример интеграции показан на рисунке \ref{img:integrated_ip}

    \begin{figure}[H]
        \centering
        \includegraphics[width=11cm]{Integrate.png}
        \caption{Пример интеграции ядра в SoC}
        \label{img:integrated_ip}
    \end{figure}

    \subsection{Возможности ядра}
    Данное ядро обладает следующими возможностями:
    \begin{itemize}
     \item поддержка интерфейсов Avalon-MM и AXI Lite;
     \item программная установка любого BaudRate, в том числе не из таблицы стандартных скоростей;
     \item программное включение бита четности;
     \item поддержка odd и even битов четности;
     \item возможность установки до четырех стоп-битов
     \item настраиваемый размер приемного и передающего буферов.
    \end{itemize}


\newpage
\section{Использование}
\subsubsection{Структура директорий}
  Проект содержит имеет следующую структуру папок:

    \dirtree{%
        .1 ..
        .2 RTL.
        .2 SIM.
        .2 OTHER.
        .2 DOC.
        .3 FIGURE.
    }

\begin{description}
    \item[\texttt{RTL}] - исходники ядра на языках Verilog и SystemVerilog,
    \item[\texttt{SIM}] - файлы для симуляции. \textbf{uart\_mm\_top\_tb.sv} - тестбенч для симуляции всего ядра. Для него предназначен \textbf{uart\_mm\_top\_wave.do} - скрипт для инициализации окна \textbf{wave} в \textbf{ModelSim/QuestaSim}). \textbf{uart\_tb.sv} - тестбенч для симуляции только приемопередатчика (проверка битов четности, стоп-битов и т.д.). Для него предназначен \textbf{wave.do} - скрипт для инициализации окна \textbf{wave} в \textbf{ModelSim/QuestaSim})
    \item[\texttt{OTHER}] - для прочих файлов,
    \item[\texttt{DOC}] - документация на ядро. Исходники в формате \LaTeX,
    \item[\texttt{DOC/FIGURE}] - хранилище изображений.
\end{description}

\newpage
\section{Архитектура}


\newpage
\section{Карта регистров}

Модуль имеет 6 регистров управления, представленные в таблице {\ref{tbl:uart_regs}}. В таблице приняты следующие обозначения: RO - Read Only (только для чтения), WO - Write Only (только для записи), WR - Write or Read (чтение и запись). Default - значение, принимаемое регистром после сброса.

\begin{table}[H]
  \begin{center}
    \begin{tabular}{l|c|c|c|l}
      \rowcolor[gray]{0.7} {\tt Имя} & {\tt Адрес, DW} & {\tt Ширина} & {\tt Доступ} & {\tt Описание} \\ \hline \hline
      {\tt CONTROL}   & 3'b000 & 30 & RW & Управление параметрами передачи.\\ \hline
      {\tt BAUD}      & 3'b001 & 32 & RW & Установка BaudRate\\ \hline
      {\tt FILL\_TX}  & 3'b010 & 32 & RO & Число байт, ожидающих передачи из TX FIFO.\\ \hline
      {\tt FILL\_RX}  & 3'b011 & 32 & RO & Число байт, ожидающих чтения из RX FIFO.\\ \hline
      {\tt TX FIFO}   & 3'b100 & 32 & WO & Запись байт в буфер передачи.\\ \hline
      {\tt RX FIFO}   & 3'b101 & 32 & RO & Чтение байт из буфера приема.\\ \hline
    \end{tabular}
    \caption{Регистры управления ядра UART}
    \label{tbl:uart_regs}
    \end{center}
\end{table}

\subsection{CONTROL}

Регистр \textbf{CONTROL} служит для управления процессами приема, передачи и контроля за флагами приемного и передающего FIFO.

\begin{table}[H]
  \begin{center}
    \begin{tabular}{l|c|c|l}
      \rowcolor[gray]{0.7}{\tt Bit} & {\tt Access} & {\tt Default} & {\tt Decription} \\\hline\hline
      {\tt [0]} & RO & 1'b1 & FIFO TX Empty \\ \hline
      {\tt [1]} & RO & 1'b0 & FIFO TX Full \\ \hline
      {\tt [2]} & RO & 1'b1 & FIFO RX Empty \\ \hline
      {\tt [3]} & RO & 1'b0 & FIFO RX Empty \\ \hline
      {\tt [7:4]} & RO & 4'b0 & Reserved \\ \hline

      {\tt [8]} & RW & 1'b0 & Enable parity bit \\ \hline
      {\tt [9]} & RW & 1'b0 & Type parity bit\\ \hline
      {\tt [11:10]} & RW & 2'b0 & Count stop bits: \\ \hline

      {\tt [12]} & RW & 1'b1 & Enable transmitter \\ \hline
      {\tt [13]} & RW & 1'b1 & Enable receiver \\ \hline
    \end{tabular}
    \caption{Биты регистра CONTROL}
    \label{tbl:control_bits}
    \end{center}
\end{table}


\textbf{FIFO TX Empty} - данный бит выставляется в единицу если FIFO для передачи не содержит никаких данных.

\textbf{FIFO TX Full} - данный бит выставляется в единицу если FIFO для передачи заполнено. Данные, записываемые в FIFO TX при установленном флаге, будут потеряны.

\textbf{FIFO RX Empty} - данный бит выставляется в единицу если FIFO RX не содержит данных для чтения.

\textbf{FIFO RX Full} - данный бит выставляется в единицу если FIFO RX заполнено. Все данные, которые будут приняты по UART будут утеряны.


\begin{table}[H]
  \begin{center}
    \begin{tabular}{ccc|l}
    P&F&T&Setting \\\hline\hline
    1 & 0 & 0 & Odd parity		\\\hline
    1 & 0 & 1 & Even parity	\\\hline
    1 & 1 & 0 & Parity bit is a Space (1'b0)\\\hline
    1 & 1 & 1 & Parity bit is a Mark (1'b1)\\\hline
    0 & & & No parity \\\hline
    \end{tabular}
  \caption{Parity setup}
  \label{tbl:parity}
  \end{center}
\end{table}




\end{document}
